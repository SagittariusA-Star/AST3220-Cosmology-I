% REMEMBER TO SET LANGUAGE!
\documentclass[landscape, a4paper,1pt,english]{article}
\usepackage{gensymb}
\usepackage{setspace}
\usepackage[utf8]{inputenc}
\usepackage[a4paper, vmargin=1pt, hmargin=2pt]{geometry}
% Standard stuff
\usepackage{amsmath,graphicx,varioref,verbatim,amsfonts}
% colors in text
\usepackage[usenames,dvipsnames,svgnames,table]{xcolor}
% Hyper refs
\usepackage[colorlinks]{hyperref}
\usepackage{float}
\usepackage{physics}
\usepackage{color}
% Document formatting
\setlength{\parindent}{0mm}
\setlength{\parskip}{1.5mm}
\usepackage{tikz}
%Color scheme for listings
\usepackage{textcomp}
\definecolor{listinggray}{gray}{0.9}
\definecolor{lbcolor}{rgb}{0.9,0.9,0.9}

%Listings configuration
\usepackage{listings}
\usepackage{subcaption}
%Hvis du bruker noe annet enn python, endre det her for å få riktig highlighting.
\lstset{
	backgroundcolor=\color{lbcolor},
	tabsize=4,
	rulecolor=,
	language=python,
        basicstyle=\scriptsize,
        upquote=true,
        aboveskip={1.5\baselineskip},
        columns=fixed,
	numbers=left,
        showstringspaces=false,
        extendedchars=true,
        breaklines=true,
        prebreak = \raisebox{0ex}[0ex][0ex]{\ensuremath{\hookleftarrow}},
        frame=single,
        showtabs=false,
        showspaces=false,
        showstringspaces=false,
        identifierstyle=\ttfamily,
        keywordstyle=\color[rgb]{0,0,1},
        commentstyle=\color[rgb]{0.133,0.545,0.133},
        stringstyle=\color[rgb]{0.627,0.126,0.941}
        }

\pagenumbering{gobble}
\usepackage{multicol}
%opening
\title{}
\author{}
\setlength{\columnsep}{0.2cm}
\setlength{\columnseprule}{0.2pt}
%\usepackage{fontspec}
%\setmainfont{Tahoma}
\begin{document}
{\setstretch{0.01} {\footnotesize {\tiny
\begin{multicols}{5}
\textcolor{red}{\textbf{S2E2a)}} Change in redshift $\dv{z}{t} = \dv{t}\frac{a(t)}{a(t_e)} = \frac{\dot{a}}{a(t_e)} + a(t)\dv{t}\frac{1}{a(t_e)} = \frac{\dot{a}}{a(t_e)} - \frac{a(t)}{a^2(t_e)}\dv{a(t_e)}{t_e}\dv{t_e}{t} = \frac{a(t)}{a(t_e)}\frac{\dot{a}}{a(t)} - \frac{a(t)}{a(t_e)}\frac{\dot{a}(t_e)}{a(t_e)}\dv{t_e}{t}$
\textcolor{blue}{\textbf{S2E2b)}} $r = \int_{t_e}^{t}\frac{cdt'}{a_0(t'/t_0)^{2/3}} = \frac{3t_0^{2/3}}{a_0}\left(t^{1/3} - t_e^{1/3}\right)\rightarrow\dv{r}{t} = 0\rightarrow t^{-2/3} - t_e^{-2/3}\dv{t_e}{t} =0\rightarrow \dv{t_e}{t} = \frac{t^{-2/3}}{t_e^{-2/3}} = \frac{1}{1+z}$
\textcolor{blue}{\textbf{S2E2c)}} Insert $\dv{t_e}{t} = \frac{1}{1+z}$ into $\dv{z}{t} = \frac{a(t)}{a(t_e)}\frac{\dot{a}}{a(t)} - \frac{a(t)}{a(t_e)}\frac{\dot{a}(t_e)}{a(t_e)}\frac{1}{1+z} = (1+z)H(t) - (1+z)H(t_e)\frac{1}{1+z} = (1+z)H(t) - H(t_e)$.
\textcolor{blue}{\textbf{S2E2d)}} $H = \frac{2}{3t}$ and $1+z = \frac{a(t)}{a(t_e)}\rightarrow t_e = \frac{t}{(1+z)^{3/2}}$ thus $\dv{z}{t}(t_0) = (1 + z)H(t_0) - H(t_e) = 5H_0 - (1+z)^{3/2}H_0 \approx -6.18 H_0$
\textcolor{blue}{\textbf{S2E2e)}} $H(t) = \frac{\dot{a}}{a} = H_0 = \text{const.}\rightarrow \dv{z}{t} = (1+z)H_0 - H_0 = (5-1)H_0 = 4H_0>0$. In dS model redshift increases constantly, in EdS it decreases constantly. $z$ measures the expansion of space between emission and reception. Neg. $\dv{z}{t}$ expansion between two times decreases, decelerating expansion.
\textcolor{red}{\textbf{S2E3a)}} $a(t) = a_0\left(\frac{t}{t_0}\right)^{3/2}\rightarrow\dot{a} = \frac{2}{3}a_0\frac{t^{-1/3}}{t_0^{2/3}}\rightarrow \ddot{a} = -\frac{2}{9}a_0\frac{t^{-4/3}}{t_0^{2/3}}\rightarrow q = -\frac{\ddot{a}a}{\dot{a}^2} = \frac{1}{2}$.
\textcolor{blue}{\textbf{S2E3b)}}$\dv{t}\frac{1}{H} = \dv{t}\frac{a}{\dot{a}} = \dv{a}{\dot{a}} + a\dv{t}\frac{1}{\dot{a}} = 1 - \frac{a\ddot{a}}{\dot{a}^2} = 1+q$.
\textcolor{blue}{\textbf{S2E3c)}} $\dv{t}\frac{1}{H} = -\frac{1}{H^2}\dv{H}{t} = -\frac{1}{H^2}\dv{H}{a}\dv{a}{t} =- \frac{\dot{a}}{H^2}\dv{H}{z}\dv{z}{a} = -\frac{\dot{a}}{H^2}\dv{a}\left(\frac{a(t_0)}{a(t)}\right)\dv{H}{z} = \frac{\dot{a}}{H_0}\left(\frac{a(t_0)}{a^2(t)}\right)\dv{H}{z} = \frac{1+z}{H}\dv{H}{z}$.
\textcolor{blue}{\textbf{S2E3d)}} Equate result from \textbf{b)} and \textbf{c)}: $\dv{H}{z}\frac{1+z}{H} = 1+q\rightarrow\ln(\frac{H(z)}{H_0}) = \int_{H_0}^{H(z)}\frac{dH'}{H'} = \int_{0}^{z}\frac{1+q(z')}{1+z'}dz'\rightarrow H(z) = H0\exp(\int_{0}^{z}\frac{1+q(z')}{1+z'}dz')$
\textcolor{blue}{\textbf{S2E3e)}} $q = q_0 =\text{const.}$: $H(z) = H_0\exp(\int_{0}^{z}\frac{1+q_0}{1+z'}dz') = H_0\exp((1+q_0)\ln(1+z)) = H_0(1+z)^{1+q_0}$.
\textcolor{red}{\textbf{S3E1a)}} Assume $a(t) = a_0 \left(\frac{t}{t_0}\right)^{2/3}$ and $k=0$. RW-metric for $d\theta = d\phi = 0$, $ds^2 = c^2dt^2 -\frac{a^2(t)dr^2}{1-kr^2} = 0$, for light. Light moving towards origin $dr<0$ for $dt >0$, then $\int_{r}^{0}dr' = -\int_{t}^{t_0}\frac{cdt'}{a(t')} = \frac{3c}{a_0}t_0^{2/3}(t_0^{1/3} - t^{1/3}) \rightarrow r = \frac{3c}{a_0}t_0\left(1-\left(\frac{t}{t_0}\right)^{1/3}\right) = \frac{3c}{a_0}t_0(1-\frac{1}{\sqrt{1+z}})  $. For $z = 6$ and $H_0  = \frac{2}{3t_0}$, $r(z = 3) = \frac{c}{a_0H_0}$. Proper dist. $d_P(t_0,z=3) = a(t_0)r = a_0\frac{c}{a_0H_0} = \frac{c}{H_0}$.
\textcolor{blue}{\textbf{S3E1b)}} Signal sent at $t_0$ and received at $t_r$. Rad. com. coord. $r=\text{const.}$, thus $r(t_0) = r(t_r)\rightarrow\frac{c}{a_0H_0} = \int_{t_0}^{t_r} = \frac{c}{a_0}t_0^{2/3}\int_{t_0}^{t_r}t^{-2/3}dt = \frac{3c}{a_0}t_0^{2/3}[t_r^{1/3} - t_0^{1/3}]\rightarrow t_r = \left[\frac{t_0^{-1}}{3H_0}+1\right]^3t_0 = \frac{27}{8}t_0$, for $H_0 = \frac{2}{3t_0}$.
\textcolor{red}{\textbf{S3E3a)}} Ansatz $a(t)=ct$, $\rho = P = 0$ and $k=-1$. $\dot{a} = c$: FI: $\dot{a}^2 + kc^2 = \frac{8\pi G}{3}\rho a^2 = 0\rightarrow c^2-c^2 = 0$ (OK!). FII: $\ddot{a} = -\frac{4\pi G}{3}(\rho+\frac{3P}{c^2})a = 0$, (OK, since $\ddot{a}$). $a(t)$ and $k=-1$ satisfy empty model.
\textcolor{blue}{\textbf{S3E3b)}} RW-metric, light rad. motion: $ds^2 = c^2dt^2 - a^2(t)\frac{dr^2}{1-kr^2}=0$. Assume $k=-1$ and $a(t) = ct$. Light moving inwards $dr<0$ when $dt>0$, thus $\frac{dr}{\sqrt{1+r^2}} = -\frac{dt}{t}\rightarrow \int_{r}^{0}\frac{dr}{\sqrt{1+r^2}} = -\int_{t}^{t_0}\frac{dt'}{t'} \rightarrow r = \sinh(\int_{t}^{t_0}\frac{dt'}{t'})$. Proper dist. when $k=-1$; $d_p(t_0) = a(t_0)\text{arcsinh}(r) = ct_0\text{arcsinh}\left[\sinh\left(\int_{t}^{t_0}\frac{dt'}{t'}\right)\right] = ct_0\int_{t}^{t_0}\frac{dt'}{t'} = ct_0\ln\frac{t_0}{t} = ct_0\ln(1+z)$. Angular diam. dist. today: $d_A(t_0) = \frac{a(t_0)r}{1+z} = \frac{ct_0}{1+z}\sinh(\ln(1+z)) = \frac{ct_0}{2(1+z)}\left(1+z - \frac{1}{1+z}\right) = \frac{c}{2H_0}\frac{z(2+z)}{(1+z)^2}$, for $H_0 = \frac{1}{t_0}$.
\textcolor{red}{\textbf{S4E1a)}} For light $ds^2 = 0$ RW-metric for $d\theta = d\phi = 0$ is $c^2dt^2 - a^2(t)\frac{dr^2}{1-kr^2} = 0$. Light moving towards origin $dr<0$ when $dt>0$. Thus $\frac{dr}{\sqrt{1-kr^2}} = -\frac{cdt}{a(t)}\rightarrow \int_{r}^{0}\frac{dr'}{\sqrt{1-kr'^2}} = -\int_{t_e}^{t_0}\frac{cdt}{a(t)}\rightarrow\int_{r}^{0}\frac{dr'}{\sqrt{1-kr'^2}} = \int_{t_0}^{t_e}\frac{cdt}{a(t)}$.
\textcolor{blue}{\textbf{S4E1b)}} Assume $r\ll1$. Since $|k|\leq 1\rightarrow |kr^2|\ll1\rightarrow 1-kr^2\approx 1$. If $(t_0-t_e)/t_0\ll1$ then the integrand $\frac{c}{a(t)}\approx\text{constant}$ so that $\int_{t_e}^{t_0}\frac{cdt}{a(t)}\approx\frac{c}{a(t_e)}\int_{t_e}^{t_0}dt = \frac{c(t_0-t_e)}{a(t_e)}$ and $\int_{0}^{r}\frac{dr'}{\sqrt{1-kr'^2}} \approx \int_{0}^{r}dr' = r$. Thus $r\approx\frac{c(t_0-t_e)}{a(t_e)}$.
\textcolor{blue}{\textbf{S4E1c)}} Taylor expand $a(t)$ around today $t_0$: $a(t) = a(t_0) + \dot{a}(t_0)(t - t_0) + \mathcal{O}(t^2)\rightarrow a(t)\approx a(t_0)[1-\frac{\dot{a}(t_0)}{a(t_0)}(t_0 - t)] = a(t_0)[1-H_0(t_0 - t)]$. At $t = t_e$ $a(t_e)\approx a(t_0)[1-H_0(t_0-t_e)]$. 
\textcolor{blue}{\textbf{S4E1d)}} Redshift of light emitted $t_e$ and received at $t_0$: $1+z = \frac{a(t_0)}{a(t_e)}\approx\frac{a(t_0)}{a(t_0)[1-H_0(t_0 - t_e)]} = \frac{1}{1-H_0(t_0-t_e)}$. Since $H_0\sim t_0\rightarrow H_0(t_0 - t_e)\sim \frac{t_0 - t_e}{t_0}\ll1$. Thus $1+z\approx\frac{1}{1-H_0(t_0-t_e)}\approx1+H_0(t_0-t_e)$ ($\frac{1}{1+x}\approx 1+x$, $x\ll1$). Thus $z\approx H_0(t_0-t_e)$.
\textcolor{blue}{\textbf{S4E1e)}} Proper distance: $d_p(t_0) = a(t_0)\mathcal{S}_k^{-1}(r)$. But for $r\ll1$ $\mathcal{S}_k^{-1}(r)\approx r$ for any $k$. Thus $d_p(t_0)\approx a(t_0)r\approx a(t_0)\frac{c(t_0-t_e)}{a(t_e} = \frac{cz}{H_0}(1+z)\approx \frac{cz}{H_0}$, for $z\ll1$. Hence $cz\approx H_0 d_p(t_0)$. 
\textcolor{blue}{\textbf{S4E1f)}} For small redshifts $z\ll1$ $d_L = a(t_0)r(1+z)\approx a(t_0)r \approx d_p(t_0)$. Hubble-Lemaitre's law $\rightarrow \dv{d_p(t)}{t} = H(t)d_p(t)\rightarrow \dv{d_p(t_0)}{t}\approx \dv{t}\frac{cz}{H_0}\approx cz\approx H_0 d_p(t_0)\approx H_0d_L(t_0)$, ($\dv{t}\frac{1}{H_0}\sim 1$ since $\frac{1}{H_0}\sim t_0$). Thus $H_0 d_L(t_0)\approx cz$, both $d_L$ and $z$ are measurable observables. So Hubble-Lemaitre's law is testable in this approximation.
\textcolor{red}{\textbf{S4E4a)}} Assume universe dominated by "phantom energy" with e.o.s. param $w=-2$. Density $\rho = \rho_0\left(\frac{a}{a_0}\right)^{-3(1+w)} = \rho_0\left(\frac{a}{a_0}\right)^3$. Energy denity increases with expanding universe!
\textcolor{blue}{\textbf{S4E4b)}} FI $k=0$: $\left(\frac{\dot{a}}{a}\right)^2 = \frac{8\pi G}{3}\rho = \frac{8\pi G}{3}\rho_0\left(\frac{a}{a_0}\right)^3 = H_0^2\frac{\rho_0}{\rho_{c0}}\left(\frac{a}{a_0}\right)^3$, ($\rho_0/\rho_{c0} = 1$, $k=0$). Thus $\dv{a}{t}\frac{1}{a(a/a_0)^{3/2}} = H_0\rightarrow \int_{a_0}^{a}\frac{da'}{a'(a'/a_0)^{3/2}} = \int_{t_0}^{t}H_0 dt'$. Let $x = \frac{a}{a_0}\rightarrow \int_{1}^{x}\frac{dx'}{x'^{5/2}} = H_0\int_{t_0}^{t}dt'\rightarrow a = a_0\left[1-\frac{3}{2}H_0(t-t_0)\right]^{-2/3}$ 
\textcolor{blue}{\textbf{S4E4c)}} When $t-t_0\to \frac{2}{3H_0}\rightarrow\frac{a}{a_0} = [1-\frac{3}{2}(t-t_0)]^{-2/3}\to[1-1]^{-2/3}\to \infty$. Space streched to $\infty$ in finit $t$. All space ripped apart, hence "Big Rip".
\textcolor{red}{\textbf{S5E1a)}} Conformal $dt = a d\eta$: Matter-dom. FI. $\left(\frac{\dot{a}}{a}\right)^2 = \frac{8\pi G}{3}\rho = \frac{8\pi G}{3}\rho_0\left(\frac{a}{a_0}\right)^{-3}\rightarrow dt = \sqrt{\frac{3}{8\pi G\rho_0}}\frac{1}{a}\left(\frac{a}{a_0}\right)^{3/2}$. Integrating conformal time $\eta = \int_{0}^{t}\frac{dt'}{a(t')} = \sqrt{\frac{3}{8\pi G \rho_0}}\frac{1}{a_0^{2/3}}\int_{0}^{a}\frac{da'}{a'^{1/2}} \propto a^{1/2}(t)$. Radi.-dom. FI: $\left(\frac{\dot{a}}{a}\right)^2 = \frac{8\pi G}{3}\rho = \frac{8\pi G}{3}\rho_0\left(\frac{a}{a_0}\right)^{-4}\rightarrow dt = \sqrt{\frac{3}{8\pi G\rho_0}}\frac{a}{a_0^2}da\rightarrow\eta = \int_{0}^{a}\frac{a'da'}{a'a_0^2}\sqrt{\frac{3}{8\pi G \rho_0}} \propto a(t)$.
\textcolor{blue}{\textbf{S5E1b)}} At matter-rad. equality $\rho_m = \rho_r\rightarrow \rho_{m0}\left(\frac{a_{eq}}{a_0}\right)^{-3}=\rho_{r0}\left(\frac{a_{eq}}{a_0}\right)^{-4}\rightarrow (1-\Omega_{m0}) = \frac{a_{eq}}{a_0}\Omega_{m0}$ (*). FI. for matter-rad.-dom. model: $\frac{H^2}{H_0^2} = \Omega_{m0}\left(\frac{a}{a_0}\right)^{-3} + \Omega_{r0}\left(\frac{a}{a_0}\right)^{-4}$. Use that $\Omega_{r0} =1-\Omega_{m0}\rightarrow dt=\frac{da}{aH_0}\left[\Omega_{m0}\left(\frac{a}{a_0}\right)^{-3} + \left(1-\Omega_{m0}\right)\left(\frac{a}{a_0}\right)^{-4}\right]^{-1/2}$. Use (*) from above and that $d\eta = \frac{dt}{a}\frac{1}{a^2H_0}\left[\Omega_{m0}\left(\left(\frac{a}{a_0}\right)^{-3} + \frac{a_{eq}}{a_0}\left(\frac{a}{a_0}\right)^{-4}\right)\right]^{-1/2}da = \frac{da}{a^2\sqrt{\Omega_{m0}H_0^2}}\left(\frac{a}{a_0}\right)^{3/2}\left[1+\frac{a_{eq}}{a_0}\left(\frac{a}{a_0}^{-1}\right)\right]^{-1/2} = \frac{da}{\sqrt{\Omega_{m0}H_0^2}}\left[a + a_{eq}\right]^{-1/2}da$ ($a_0 = 1$).  Integrating $\eta = \int_{0}^{\eta}d\eta' = \int_{0}^{a}\frac{da'}{\sqrt{\Omega_{m0}H_0^2}}\left[a'+a_{eq}\right]^{-1/2}da' = \frac{1}{\sqrt{\Omega_{m0}H_0^2aa_{eq}}}\int_{0}^{a}\frac{da'}{\sqrt{1+\frac{a'}{a_{eq}}}} = \sqrt{\frac{a_{eq}}{\Omega_{m0} H_0^2}}\int_{1}^{1+a/a_{eq}}\frac{du}{\sqrt{u}} = \frac{2\sqrt{a_{eq}}}{\sqrt{\Omega_{m0}H_0^2}}\left(\sqrt{1 + \frac{a}{a_{eq}}} - 1\right) = \frac{2}{\sqrt{\Omega_{m0}H_0^2}}\left(\sqrt{a + a_{eq}} - \sqrt{a_{eq}}\right)$.
\textcolor{red}{\textbf{S5E2a)}} Dust, $\Lambda$ and $k\neq0$, FI: $\frac{\dot{a}^2}{a^2} +\frac{kc^2}{a^2} = \frac{8\pi G}{3}\left(\rho_{m0}(1+z)^3 + \rho_{\Lambda0}\right) = H_0^2\frac{8\pi G}{3H_0^2}\left(\rho_{m0}(1+z)^3 + \rho_{\Lambda0}\right)$. Crit. density today $\rho_{c0} = \frac{3H_0^2}{8\pi G}$ thus $H^2(z) = H_0^2\left(\Omega_{m0}(1+z)^3 + \Omega_{\Lambda0}\right) - \frac{kc^2}{a_0^2H_0^2}(1+z)^2H_0^2 = H_0^2\left(\Omega_{m0}(1+z)^3 + \Omega_{\Lambda0} + \Omega_{k0}(1+z)^2\right)$ (for $\Omega_{k0} = -\frac{kc^2}{a_0^2H_0^2}$). Then FII: $\ddot{a} = -\frac{4\pi G}{3}\left(\rho_{m0}(1+z)^3 + \rho_{\Lambda0} - 3P_m - 3P_\Lambda\right)a = -\frac{8\pi G}{3H_0^2}\frac{H_0^2}{2}\left(\rho_{m0}(1+z)^3 - 2\rho_{\Lambda0}\right)a\rightarrow \frac{\ddot{a}}{a} = -\frac{H_0^2}{2}\left(\Omega_{m0}(1+z)^3 - 2\Omega_{\Lambda0}\right)$. FI: must hold for all $t$ or $z$, so FI($z = 0$) must hold for all $z$: $H^2(z=0) = H_0^2 = H_0^2\left[\Omega_{m0} + \Omega_{k0} + \Omega_{\Lambda0}\right]\rightarrow \Omega_{m0} + \Omega_{k0} + \Omega_{\Lambda0} = 1$ for all $z$.
\textcolor{blue}{\textbf{S5E2b)}} For "Bounce" the min. $z_*$ must fulfil $\dot{a}(z_*) = 0$ and $\ddot{a}(z_*)\geq 0$ for extremal to be minimum.
\textcolor{blue}{\textbf{S5E2c)}} At min. $\dot{a}(z_*) = 0\rightarrow H^2(z_*) = 0\rightarrow \Omega_{m0}(1+z_*)^3 + \Omega_{k0}(1+z_*)^2 + \Omega_{\Lambda0} = \Omega_{m0}(1+z_*)^3 + (1-\Omega_{m0}-\Omega_{\Lambda0})(1+z_*)^2 + \Omega_{\Lambda0}= 0$ using $\sum \Omega_{i0} = 1$. Thus $\Omega_{\Lambda0} = \frac{(1+z_*)^2(\Omega_{m0}z_* + 1)}{z_*(2 + z_*)}$. Second condition: $\ddot{a}(z_*)\geq0\rightarrow -\frac{H_0^2}{2}[\Omega_{m0}(1+z_*)^3 - 2\Omega_{\Lambda0}]\geq 0$. Insert for $\Omega_{\Lambda0}$ from first condition: $\Omega_{m0}(1+z_*)^3 - \frac{2(1+z_*)^2(\Omega_{m0}z_* + 1)}{z_*(2+z_*)}\leq0\rightarrow \Omega_{m0}\leq\frac{2}{z_*^2(z_*+3)}$.
\textcolor{blue}{\textbf{S5E2d)}} Quasar with redshift $z=6$ observed. Upper limit on $\Omega_{m0}$: $z_*\geq z = 6\rightarrow\Omega_{m0}\leq \frac{2}{z_*^2(3+z_*)}\leq\frac{2}{6^2\cdot9}\approx6\cdot10^{-3}$. Most observations indicate $\Omega_{m0}\approx0.3$. Hence this model is likely not describing our universe.
\textcolor{red}{\textbf{S5E3a)}} Rad.-$\Lambda$ equality: $\rho_r = \rho_{\Lambda}\rightarrow\rho_{r0}\left(\frac{a_{eq}}{a_0}\right)^{-4} = \rho_{\Lambda0}\rightarrow a_{eq} = a_0\left(\frac{\rho_{r0}}{\rho_{\Lambda0}}\right)^{1/4} =a_0\left( \frac{\rho_{r0}/\rho_{c0}}{\rho_{\Lambda0}/\rho_{c0}}\right)^{1/4} = a_0\left(\frac{\Omega_{r0}}{\Omega_{\Lambda0}}\right)^{1/4}$.
\textcolor{blue}{\textbf{S5E3b)}} FI: $\frac{\dot{a}^2}{a^2} = H^2 = \frac{8\pi G}{3}(\rho_{r0}\left(\frac{a}{a_0}\right)^{-4} + \rho_{\Lambda0}) = H_0^2(\Omega_{r0}\left(\frac{a}{a_0}\right)^{-4} + \Omega_{\Lambda0})$.
\textcolor{blue}{\textbf{S5E3c)}} \textbf{b)}$\rightarrow\dv{a}{t}\frac{1}{a}= H_0\sqrt{\frac{\Omega_{r0}}{a^4} + \Omega_{\Lambda0}} = \frac{H_0\sqrt{\Omega_{r0}}}{a^2}\sqrt{1+\frac{a^4}{a_{eq}^4}}$. Integrating $t = \int_{0}^{t}dt' = \frac{1}{H_0\sqrt{\Omega_{r0}}}\int_{0}^{a}\frac{a'da'}{\sqrt{1+\frac{a'^4}{a_{eq}^4}}}$.
\textcolor{blue}{\textbf{S5E3c)}} Let $x = \left(\frac{a}{a_{eq}}\right)^2$, then $t = \frac{1}{H_0\sqrt{\Omega_{r0}}}\int_{0}^{a}\frac{a'da'}{\sqrt{a+\frac{a'^4}{a_{eq}^4}}} = \frac{a_{eq}^2}{2H_0\sqrt{\Omega_{r0}}}\int_{0}^{x}\frac{dx'}{\sqrt{1+x'}} = \frac{a_{eq}^2}{2H_0\sqrt{\Omega_{r0}}}\text{arcsinh}(x)$. Thus at $t = t_{eq} = \frac{\text{arcshinh}(1)}{2H_0\sqrt{1-\Omega_{r0}}}\approx4.2$Gyr. 
\textcolor{blue}{\textbf{S5E3e)}} Trasition from decel. to acc. when $\ddot{a} = 0$. FII: $\frac{\ddot{a}}{a} = -\frac{4\pi G}{3}\left(\frac{\rho_{r0}}{a^4} +\rho_{\Lambda0} +\frac{3P_r}{c^2} +\frac{3P_\Lambda}{c^2}\right) = -\frac{4\pi G}{3}(\frac{2\rho_{r0}}{a^4} - 2\rho_{\Lambda0}) = -H_0^2\left(\frac{\Omega_{r0}}{a^4} - \Omega_{\Lambda0}\right)$. At $\ddot{a}=0 \rightarrow a_{acc} = \left(\frac{\Omega_{r0}}{\Omega_{\Lambda0}}\right)^{1/4} = a_{eq}$ which we found was at $t = t_{eq}\approx9.2$ Gyr in \textbf{d)}.
\textcolor{red}{\textbf{S5E4a}} Model with $k=0$ and $a(t) = Ct^2\rightarrow\dot{a} = 2ct^2\rightarrow \ddot{a} ct>0$ so acc. expansion. Hubble param. $H\equiv\frac{\dot{a}}{a} = \frac{2}{t}$ decreasing with time. 
\textcolor{blue}{\textbf{S5E4b)}} dS model $a(t) = a_0\exp(H_0(t-t_0))\rightarrow \dot{a} = H_0a\rightarrow\ddot{a} = H_0^2a >0$ has acc. expansion. Hubble param. $\rightarrow H = H_0=\text{const.}>0$. So \textbf{a)} and \textbf{b)} show that increasing $H$ does not always mean acc. expansion.
\textcolor{blue}{\textbf{S5E4c)}} $k=0$ and $\rho + \frac{P}{c^2}$: FI: $H^2\left(\frac{\dot{a}}{a}\right)^2 = \frac{8\pi G}{3}\rho$ and FII: $\frac{\ddot{a}}{a} = -\frac{4\pi G}{3}(\rho + \frac{3P}{c^2})$. Consider $\dv{H}{t} = \dv{t}\frac{\dot{a}}{a} = \frac{\ddot{a}}{a}-\frac{\dot{a}^2}{a^2} = -\frac{4\pi G}{3}(\rho + \frac{3P}{c^2})-\frac{8\pi G}{3}\rho = -4\pi G(\rho + \frac{P}{c^2}) <0$, since $\rho + \frac{P}{c^2}>0$ by assup. QED. 
\textcolor{red}{\textbf{S6E1a)}} In general the lum. dist: $d_L = \frac{c(1+z)}{H_0\sqrt{|\Omega_{k0}|}}\mathcal{S}_k\left(\sqrt{|\Omega_{k0}|}\int_{0}^{z}\frac{dz}{H(z)/H_0}\right) = \frac{c(1+z)}{H_0}\int_{0}^{z}\frac{dz}{H(z)/H_0}$, since flat $k=0$. FI for dust and $\Lambda$: $\left(\frac{H}{H_0}\right)^2 = \Omega_{m0}(1+z)^3 + \Omega_{\Lambda0}\rightarrow \frac{H(z)}{H_0} = \sqrt{\Omega_{m0}(1+z) + 1 - \Omega_{m0}}$, since $\Omega_{m0} + \Omega_{\Lambda0} = 1$.Thus $d_L = \frac{c(1+z)}{H_0}\int_{0}^{z}\frac{dz}{\sqrt{\Omega_{m0}(1+z)^3 + 1 - \Omega_{m0}}}$.
\textcolor{blue}{\textbf{S6E1b)}} \textbf{i)} Let $\Omega_{m0} = 1$, then $d_L = \frac{c(1+z)}{H_0}\int_{0}^{z}\frac{dz}{\sqrt{(1+z)^3}} = \frac{c(1+z)}{H_0}\int_{1}^{1+z}u^{-3/2}du = \frac{2c}{H_0}\left(1+z - \sqrt{1+z}\right)$.
\textbf{ii)} Let $\Omega_{m0} = 0$, then $d_L = \frac{c(1+z)}{H_0}\int_{0}^{z}dz = \frac{c(1+z)z}{H_0}$.
\textcolor{blue}{\textbf{S6E1c)}} For $z\ll1\rightarrow z'\ll1$ ($z'\leq z$ in integral). $\rightarrow \Omega_{m0}(1+z')^3 + 1- \Omega_{m0}\approx\Omega_{m0}(1+3z') + 1 - \Omega_{m0} = 1+3\Omega_{m0}z'$ thus $d_L \approx \frac{c(1+z)}{H_0}\int_{0}^{z}(1+3\Omega_{m0}z')^{-1/2}dz'\approx\frac{c(1+z)}{H_0}\int_{0}^{z}(1-\frac{3}{2}\Omega_{m0}z')dz' = \frac{c(1+z)}{H_0}(z-\frac{3}{4}z'^2) = \frac{c}{H_0}(z -\frac{3}{4}z^2 + z^2 - \frac{3}{4}z^3) = \frac{cz}{H_0}+\mathcal{O}(z^2) + \mathcal{O}(z^3)\approx\frac{cz}{H_0}$.
\textcolor{blue}{\textbf{S6E1d)}} Assume $M_{97} = M_{92}$ then $-5\log(\left(\frac{d_L^{97}}{10pc}\right)) +m_{97}= -5\log(\left(\frac{d_L^{92}}{10pc}\right)) +m_{92}\rightarrow d_L^{97} = d_L^{92}10^{\frac{1}{5}(m_{97}-m_{92})}$. Since $z_{92} = 0.026\ll1\rightarrow d_L^{92} \approx\frac{c z_{92}}{H_0}$, thus $d_L^{97}\approx1.16\frac{c}{H_0}$.
\textcolor{blue}{\textbf{S6E1e)}} \textbf{i)} Dust only $\Omega_{m0} = 1\rightarrow d_L \approx 0.95\frac{c}{H_0}$ ($z = 0.83$) too low $d_L$. \textbf{ii)} $\Lambda$ only $\Omega_{m0} = 0\rightarrow d_L \approx1.52\frac{c}{H_0}$ too high $d_L$. Right mix of dust and $\Lambda$ may give good result.
\textcolor{red}{\textbf{S6E3}} FII: $\frac{\ddot{a}}{a} = -\frac{4\pi G}{3}\left(\rho_m + \rho_\Lambda + \frac{3P_m}{c^2} + \frac{3P_\Lambda}{c^2}\right) = -\frac{H_0^2}{2}\left(\Omega_{m0}\left(\frac{a_0}{a}\right)^3 - 2\Omega_{\Lambda0}\right)$. $\Omega_{m0}$: tries to make $\ddot{a}<0$, $a(t)$ curves down. $\Omega_{m0}$: tries to make $\ddot{a}>0$, $a(t)$ curves up. $H_0$ fixed $\rightarrow$ slope of $a(t)$ fixed at $t=t_0$. (We choose $t=0$ where $a(t)=0$ in past). Making $\ddot{a}$ more negative , makes $a(t)=0$ at earlier in the past. Making $\ddot{a}$ more positive , makes $a(t)=0$ at later in the past.  
\textcolor{red}{\textbf{S7E1}} Assume dominated by $\rho\propto a^{-\alpha}$, then FI:
$H^2\propto\rho\propto a^{-\alpha}(t)\rightarrow \dv{a}{t}\propto a^{1-\frac{\alpha}{2}}\rightarrow d_p^{PH} = a(t)\int_{0}^{t}\frac{cdt}{a(t)}\propto\int_{0}^{t}\frac{dt}{a(t)} \propto \int_{0}^{t}a^{\frac{\alpha}{2} - 2}da$. If $\frac{\alpha}{2} - 2 = -1$ $d_p^{PH}\to-\infty$ and $d_p^{PH} \propto \frac{1}{\frac{\alpha}{2} - 1}\left[a^{\frac{\alpha}{2} - 1}\right]_0^a\to a^{\frac{\alpha}{2} - 1}$ if $\frac{\alpha}{2}-1\geq0$ and $\to-\infty$ if $\frac{\alpha}{2}-1<0$. Thus $\alpha \geq 2$.
\textcolor{red}{\textbf{S7E2a)}} EdS: $a(t)a_0(\frac{t}{t_0})^{2/3}\rightarrow t = t_0 (\frac{a}{a_0})^{3/2} = t_0(1+z)^{-3/2}$
\textcolor{blue}{\textbf{S7E2b)}} $t(z) = t_0(1+z)^{-3/2} = \frac{2}{3H_0}(1+z)^{-3/2}\approx0.5$ Gyr.
\textcolor{blue}{\textbf{S7E2c)}} Open $k=-1$ model dust only: $\frac{a(u)}{a_0} = \frac{1}{2}\frac{\Omega_{m0}}{1-\Omega_{m0}}(\cosh(u) - 1)$ and $t(u) = \frac{\Omega_{m0}}{2H_0(1-\Omega_{m0})^{3/2}}(\sinh u - u)$. $u = \frac{a}{a_0} = \frac{1}{1+z} = \frac{1}{2}\frac{\Omega_{m0}}{1-\Omega_{m0}}(\cosh u - 1)\rightarrow u = \cosh^{-1}\left(\frac{5}{3}\right)\rightarrow t(z = 6) \approx 0.84$Gyr.
\textcolor{blue}{\textbf{S7E2d)}} Flat $\Lambda$CDM: $H_0 t = \frac{2}{3\sqrt{1-\Omega_{m0}}}\sinh^{-1}\left[\left(\frac{a}{a_{m\Lambda}}\right)^{3/2}\right]$, for $\frac{a_{m\Lambda}}{a_0} = \left(\frac{\Omega_{m0}}{1-\Omega_{m0}}\right)^{1/3}$ thus $t(z = 6) \approx 0.89$ Gyr.
\textcolor{blue}{\textbf{S7E2e)}} $a(t) = (t/t_0)^\alpha\rightarrow\dot{a} = \alpha r^{\alpha - 1}/t_o^\alpha \rightarrow \ddot{a} = \alpha(\alpha - 1)\frac{t^{\alpha-2}}{t_0^\alpha}$. Acc. rate $\ddot{a}>0\rightarrow \alpha(\alpha - 1)>0\rightarrow\alpha>1$. Consider $r = \int_{t_1}^{t_2}\frac{cdt}{a(t)} = \int_{t_1}^{t_2}\frac{cdt}{(t/t_0)^\alpha} = \frac{ct_0^\alpha }{1-\alpha}[t^{1-\alpha}]_{t_1}^{t_2}$ (finit horizon). $t_1\to0$, $t_2$ fixed $\rightarrow$ P.H. if $\alpha<1$. $t_2\to\infty$, $t_1$ fixed $\rightarrow$ E.H. if $\alpha>1$. $\alpha = 0\rightarrow r\rightarrow r\sim\int dt/t = \ln t$ diverge. E.H. and acc. expansion if $\alpha >1$.
\textcolor{blue}{\textbf{S7E2f)}} For $a(t) = (t/t_0)^\alpha\rightarrow H = \frac{\alpha t^{\alpha -1}/t_0^\alpha}{t^\alpha/t_0^\alpha} = \frac{\alpha}{t}$. Found for this model acc. expansion if $\alpha >1$, but then $H = \frac{\alpha}{t}>0$ and is decreasing. Statement not true.
\textcolor{blue}{\textbf{S7E2g)}} $aH = a\frac{\dot{a}}{a} = \dot{a}\rightarrow \dv{aH}{t} = \dv{\dot{a}}{t} = \ddot{a}>0$, by def. of acc. expansion, $aH$ increasing with time.
\textcolor{red}{\textbf{S7E3a)}} Rad. energy density $\rho c^2\propto T^4\propto a^{-4}\rightarrow T\propto a^{-1}\propto(1+z)$. $T_0 \propto 1+z_0 = 1\rightarrow \frac{T_0}{T_{dec}} = \frac{1}{1+z_{dec}}\rightarrow z_{dec} = \frac{T_{dec}}{T_0} - 1 \approx 1098$.
\textcolor{blue}{\textbf{S7E3b)}} FI for $\Omega_{m0} = 1$, $k=0$: $\frac{H^2}{H_0^2} = \Omega_{m0}\left(\frac{a}{a_0}\right)^{-3}\rightarrow \dv{a}{t}\frac{1}{a} = H_0\left(\frac{a}{a_0}\right)^{-3/2}\rightarrow t_0 = \int_{0}^{t_0} dt = \frac{1}{H_0}\int_{0}^{a_0}\frac{\sqrt{a}}{a_0^{3/2}}da = \frac{2}{3H_0}\approx 9.3$Gyr.
\textcolor{blue}{\textbf{S7E3c)}} Same integral as in \textbf{b)}: $t_{dec} = \int_{0}^{t_{dec}} = \frac{1}{H_0}\int_{0}^{a_{dec}}\frac{\sqrt{a}}{a_0^{3/2}}da = \frac{2}{3H_0}\left(\frac{a_{dec}}{a_0}\right)^{3/2} = t_0(1+z_{dec})^{-3/2}\approx 2.5\cdot 10^5$yr.
\textcolor{blue}{\textbf{S7E3d)}} P.H. $d_{PH} = a_{dec}r = a_{dec}\int_{0}^{t_{dec}}\frac{c dt}{a(t)}= a_{dec}\int_{0}^{t_{dec}}\frac{c}{a(t)}\frac{\sqrt{a(t)}}{H_0a_0^{3/2}}da = \frac{2a_{dec}}{a_0^{3/2}}\frac{c}{H_0}\sqrt{a_{dec}} = \frac{2c}{H_0}(1+z_{dec})^{-3/2}\approx0.235$Mpc.
\textcolor{blue}{\textbf{S7E3e)}} Proper distance to $z_{dec}$: $d_{p} = a_{0}r(z_{dec}) = a_{0}\int_{t_{dec}}^{t_{0}}\frac{c dt}{a(t)}= \frac{a_0}{a_0^{3/2}}\frac{c}{H_0}\int_{a_{dec}}^{a_0}\frac{da}{\sqrt{a}} = \frac{2c}{H_0}\frac{1}{\sqrt{a_0}}(\sqrt{a_0} - \sqrt{a_{dec}}) = \frac{2c}{H_0}(1-\sqrt{\frac{a_{dec}}{a_0}}) = \frac{2c}{H_0}(1-\frac{1}{\sqrt{1+z_{dec}}})\approx8.3\cdot10^3$Mpc.  
\textcolor{blue}{\textbf{S7E3f)}} Angular diameter distance:
$d_A(z_{dec}) = \frac{d_{ph}}{\theta_{ph}} = \frac{a(t_{dec})}{a_0}a_0r(z_{dec})=\frac{a_0r(z_{dec})}{1+z_{dec}}\rightarrow\theta_{ph} = d_{ph}(z_{dec})\frac{1+z_{dec}}{a_0 r(z_{dec})}\approx \frac{0.24Mpc(1+1098)}{8.3\cdot10^3 Mpc}\approx0.032 rad \approx 1.8^\circ$.
\textcolor{red}{\textbf{S7E4a)}} EdS, $k = 0$, $\Omega_{m0}=1$: $a(t) = a_0\left(t/t_0\right)^{2/3}\rightarrow r_{PH}(t_0) = \int_{0}^{t_0}\frac{cdt}{a(t)} = \int_{0}^{t_0}\frac{cdt}{a_0(t/t_0)^{2/3}} = \frac{3c}{a_0}t_0 = \frac{2c}{H_0}$, $a_0 = 1$. E.H. in dS: $a(t) = a_0\exp[H_0(t - t_0)]\rightarrow r_{EH}(t_0) = \int_{t_0}^{\infty}\frac{cdt}{a(t)} = \frac{c}{a_0}\int_{t_0}^{\infty} \exp([-H_0(t-t_0)])dt = -\frac{c}{a_0H_0}\left[\exp[-H_0(t-t_0)]\right]_{t_0}^\infty = \frac{c}{H_0}$, $a_0 = 0$.
\textcolor{blue}{\textbf{S7E4b)}} Light emitted at $t$ and received today has rad. comoving coord. $r = \int_{t}^{t_0}\frac{cdt}{a(t)} = \frac{c}{a_0}\int_{t}^{t_0} \exp([-H_0(t-t_0)])dt = \frac{c}{a_0H_0}[\exp([-H_0(t-t_0)]) - 1]$. Redshift $1+z = \frac{a_0}{a_0} = \exp(-H_0(t-t_0))\rightarrow
r = \frac{c}{a_0H_0}(1+z-1) = \frac{cz}{a_0H_0}$. Thus $r_{EH} = r(z = 1) \rightarrow z = 1$.
\textcolor{blue}{\textbf{S7E4c)}} EdS: Teacher treated as light: $r_{PH} = \int_{t_0}^{t_f}\frac{cdt}{a(t)} = \frac{c}{a_0}\int_{t_0}^{t_f}\frac{cdt}{(t/t_0)^{2/3}} = \frac{3c}{a_0}t_0^{2/3}[t_f^{1/3} - t_0^{1/3}] = \frac{2c}{a_0H_0}\left[\left( \left(\frac{t_f}{t_0}\right)^{1/3}-1\right)\right]$. Found in \textbf{a)} that $r_{PH} = \frac{2c}{a_0H_0}$ for EdS. $\rightarrow\frac{2c}{a_0H_0}\left[ \left(\left(\frac{t_f}{t_0}\right)^{1/3}-1\right) \right] = \frac{2c}{a_0H_0}\rightarrow t_f = 8t_0 = \frac{16}{3H_0}$.
\textcolor{blue}{\textbf{S7E4d)}} dS model: $r_{EH} =\frac{c}{a_0H_0} = \int_{t_0}^{t_f}\frac{cdt}{a(t)} = \frac{c}{a_0H_0}[1 - \exp([-H_0(t_f - t_0)])]$, which cannot be fulfilled for finit $t_f\rightarrow t_f = \infty$.
\textcolor{blue}{\textbf{S7E4e)}} Dust and vacuum energy: $\Omega_{m0} + \Omega_{\Lambda0} = 1$. FI: $H^2 = H_0^2(\Omega_{m0}\left(\frac{a}{a_0}\right)^{-3} + \Omega_{\Lambda0}) = H_0^2(\Omega_{m0}\left(\frac{a}{a_0}\right)^{-3} + 1-\Omega_{m0})\rightarrow\dv{a}{t}\frac{1}{a} = H_0\sqrt{\Omega_{m0}\left(\frac{a}{a_0}^{-3} + 1 - \Omega_{m0}\right)}$. Redshift $1+z = \frac{a_0}{a}\rightarrow\dv{z}{a} = -\frac{a_0}{a^2}$ gives $dt = -\frac{(1+z)^{-1}dz}{H_0\sqrt{\Omega_{m0}(1+z)^3 + 1 - \Omega_{m0}}}$. Rad. comoving coord. for E.H. $r_{EH} = \int_{t_0}^{\infty}\frac{cdt}{a(t)} = \frac{c}{a_0H_0}\int_{0}^{-1} -\frac{dz}{\sqrt{\Omega_{m0}(1+z)^3 +1 - \Omega_{m0}}} = \frac{c}{a_0 H_0}\int_{-1}^{0}\frac{dz}{\sqrt{\Omega_{m0}(1+z)^3 + 1 - \Omega_{m0}}}$ and for P.H. $r_{PH} = \int_{0}^{t_0}\frac{cdt}{a(t)}=\frac{c}{a_0H_0}\int_{\infty}^{0}-\frac{dz}{\sqrt{\Omega_{m0}(1+z)^3 + 1 -\Omega_{m0}}} = \frac{c}{a_0H_0}\int_{0}^{\infty}\frac{dz}{\sqrt{\Omega_{m0}(1+z)^3 + 1 -\Omega_{m0}}}$
\textcolor{blue}{\textbf{S7E4f)}} For $0<\Omega_{m0}<1$ integrand goes as $(1+z)^{3/2}\sim x^-p$, which makes the integral convergent for $p>1$. 
\textcolor{red}{\textbf{S8E1a)}} Confined to lin. dim. R, Heisenberg $\rightarrow p\sim \frac{\hbar}{R}\rightarrow E = pc \sim \frac{\hbar c}{R}\propto \frac{1}{R}$.
\textcolor{blue}{\textbf{S8E1b)}} At $R\to0$ $E_H\sim\frac{C}{R}$ for $R\to\infty$ $E_H\sim R^3\to$ min at $\dv{E_H}{R} = 4\pi BR^2 - \frac{C}{R^2} = 0\rightarrow C = 4\pi BR^4$. Insert $C$ to find $\min{E_H}$ 
\textcolor{blue}{\textbf{S8E1c)}} Set $E_{H,min} = E_0\sim 10^{3}$MeV and $R = 1$fm. $\frac{16\pi}{3}R^3 B \approx 10^3$ MeV $\rightarrow B \approx 59.7$ MeV/fm$^3$.
\textcolor{blue}{\textbf{S8E1d)}} $[B]_{SI} = \frac{MeV}{fm^3}=[B]_{nat}[\hbar c]_{SI}^\alpha\rightarrow\frac{[B]_{SI}}{B_{nat}} = (MeVfm)^{-3}\rightarrow = \alpha = -3\rightarrow B = B_{nat}(\hbar c)^{-3} = (200MeV)^4(197.327MeVfm)^{-3}\approx = 208.24 MeVfm^{-3}\rightarrow R\approx\left(\frac{3}{16\pi}\frac{E_{min}}{B}\right)^{1/3} \approx0.66$fm, for $E_{min} = E_0 = 10^3$MeV.
\textcolor{blue}{\textbf{S8E1e)}} 1. law of Therm.dyn. $dE = TdS - PdV + \nu dN\rightarrow P =- \left(\dv{E}{V}\right)_S = -\left(\dv{BV}{V}\right)_S = -B$. Hadronic phase: 3 hadrons, all spin 0 thus 1 int. d.o.f. $\pi^+$ anti-part of $\pi^-$, $\pi^0$ is own anti-part. No anti-part. contribution $\rightarrow g_0 = 3$. Pressure thus $P_H = \frac{\pi^2}{90}3\frac{(k_BT)^4}{\hbar c}^3 = \frac{\pi^2}{30}\frac{(k_BT)^4}{\hbar c}^3$. QCD phase: 8 gluons (2 polar.), u,d quark (spin-$\frac{1}{2}$ ferm), 2 spins, 3 color, 1 anti-part. Equil. with phonos (2 polar.) $\rightarrow g_* = 8\cdot2  \frac{7}{8}(2\cdot2\cdot2\cdot3) = 37$. Pressure $P_{QCD} = \frac{37\pi^2}{90}\frac{(k_BT)^4}{(\hbar c)^3} - B$, now bag contribution $-B$.
\textcolor{blue}{\textbf{S8E1f)}} $P_H = P_{QCD}\rightarrow \frac{\pi^2}{30}\frac{(k_BT_C)^4}{(\hbar c)^3} = \frac{37\pi^2}{90}\frac{(k_BT_C)^4}{(\hbar c)^3} - B\rightarrow k_BT_C = \left[\frac{45}{17\pi^2}(\hbar c)^2 B\right]^{1/4} \approx140$MeV.
\textcolor{blue}{\textbf{S8E1g)}} $t\approx 2.423 g_*^{-1/2}\left(\frac{k_B T}{1MeV}\right)^{-2}$s. Before QCD phase-transs, quarks and gluons give $g_*=37$. Also $k_BT \geq 140$MeV, photons, electrons, muons and neutrinos relat. and in equil. $\rightarrow g_* = 37 + 2 + \frac{7}{8}(2\cdot2\cdot + 2\cdot2 + 3\cdot2\cdot1) = \frac{205}{4}\rightarrow t\approx 1.7\cdot10^{-5}$s.
\textcolor{red}{\textbf{S9E2a)}}: All $\nu$ have momentum $p$ thus the same energy $E = \sqrt{p^2 + m^2}$. $\rho = \frac{NE}{V} = n\sqrt{p^2+m^2}$.
\textcolor{blue}{\textbf{S9E2b)}} Pressure $P = \frac{1}{3}\frac{np}{\sqrt{1+(\frac{m}{p})^2}} = w\rho = wn\sqrt{p^2+m^2}\rightarrow w = \frac{1}{3}\frac{1}{1 + (\frac{m}{p})^2}$
\textcolor{blue}{\textbf{S9E2c)}} $p = p_0/a\rightarrow w(a) = \frac{1}{3}\frac{1}{1 + (\frac{m}{p_0/a})^2} = \frac{1}{3}\frac{1}{1 + \frac{a^2}{f_0^2}}$, for $f_0 = \frac{p_0}{m}$. 
\textcolor{blue}{\textbf{S9E2d)}} Cont. eq.: $\dot{\rho} = -3\frac{\dot{a}}{a}(\rho + P) = -3\frac{\dot{a}}{a}(1+w(a))\rho\rightarrow\int_{\rho_0}^{\rho}\frac{d\rho'}{\rho'} = -3\int_{a_0}^{a}\frac{1+w(a)}{a}da = -3\ln\frac{a}{a_0}-3I_w\rightarrow\rho = \rho_0a^{-3}\exp(-3I_w(a))$
\textcolor{blue}{\textbf{S9E2e)}} The integral ($a_0=1$) $-3I_w(a) = -3\int_{a_0}^{a}\frac{w(a)}{a}da = -\int_{a_0}^{a}\frac{da}{a(1+\frac{a^2}{f_0^2})} = -f_0^2\int_{a_0}^{a}\frac{da}{a(f_0^2 + a^2)} = -\int_{a_0}^{a}(\frac{1}{a} - \frac{a}{f_0^2 + a^2} )da = -\ln\frac{a}{a_0} + \frac{1}{2}\ln\frac{f_0^2 + a^2}{f_0^2 + a_0^2} = \ln\sqrt{\frac{f_0^2/a^2 + 1}{f_0^2 + 1}}$ which gives $\rho = \rho_0 a^{-3}\exp(-3I_w) = \rho_0a^{-3}\sqrt{\frac{1 + f_0^2a^{-2}}{1 + f_0^2}} = \rho_0g^{1/2}(a)a^{-3}$.
\textcolor{blue}{\textbf{S9E2f)}} FI and $k=0$: $\frac{\dot{a}^2}{a^2} = \frac{8\pi G}{3}\rho = H_0^2 \frac{8\pi G}{3H_0^2}\rho_0 g^{1/2}(a)a^{-3} = H_0^2 g^{1/2}a^{-3}$ since $\frac{\rho_0}{\rho_{c0}} = 1$ for $k=0$. Thus $\frac{\sqrt{a}da}{g^{1/4}} = H_0 dt$ giving $\int_{0}^{t}H_0 dt = H_0 t = \int_{0}^{a}\frac{\sqrt{a}da}{g^{1/4}}$. Calculating the integral $\int_{0}^{a}\frac{\sqrt{a}da}{g^{1/4}}=\int_{0}^{a}\frac{\sqrt{a}da}{1+f_0^2a^{-2}}(1+f_0^2)^{1/4} = (1+f_0^2)^{1/4}\int_{0}^{a}\left(\frac{a}{1+f_0^2a^{-2}}\right)^{1/4}da = (1+f_0^2)^{1/4}\int_{0}^{a}\frac{ada}{(a^2 + f_0^2)^{1/4}}$. Substitute $u = a^2+f_0^2$  giving $= (1+f_0^2)^{1/4}\int_{f_0^2}^{a^2 + f_0^2}\frac{du}{u^{1/4}} = \frac{2}{3}(1+f_0^2)^{1/4}((a^2 + f_0^2)^{3/4} - f_0^{3/2})$.
\textcolor{blue}{\textbf{S9E2g)}} Consider $w(a)$. When $f_0\to0\rightarrow w(a)\to0$ (massive neutrinos, dust). For $f_0\to\inf\rightarrow w(a)to \frac{1}{3}$ (massless neutrinos, radiation). Next consider $H_0t$ from previous task. For $f_0\to 0$: $H_0t\to\frac{2}{3}a^{3/2}\rightarrow a = \left(\frac{3H_0}{2}t\right)^{2/3} = \left(\frac{t}{t_0}\right)^{2/3}$ (EdS). Next let $f_0\to\infty\rightarrow f_0^2 + 1\approx f_0^2$. Then $H_0t\to \frac{2}{3}f_0^{1/2}\left[f_0^{3/2}(1+\left(\frac{a}{f_0}\right)^2)^{3/4} - f_0^{3/2}\right] = \frac{2}{3}f_0^2\left[\left(1 + \left(\frac{a}{f_0}\right)^2\right)^{3/4} - 1\right] \approx \frac{2}{3}f_0^2\left[1 + \frac{3}{4}\left(\frac{a}{f_0}\right)^2 - 1\right] = \frac{1}{2}a^2\rightarrow a = (2H_0t)^{1/2} = (t/t_0)^2$ (rad. dom. flat universe). Here we used that $\left(1+x\right)^{3/4}\approx1+\frac{3}{4}x$ for $x\ll1$.
\textcolor{blue}{\textbf{S9E2h)}} Solve $H_0t = \frac{2}{3}(1+f_0^2)^{1/4}\left[\left(f_0^2 + a^2\right)^{3/4} - f_0^{3/2}\right]$ for $a$: $a = \left[\left(\frac{3}{2}8(1+f_0)^{-1/4}H_0t + f_0^{3/2}\right)^{4/3} - f_0^2\right]^{1/2} = \left[\left[\frac{t}{t_{fid}} + f_0^{3/2}\right]^{4/3} -f_0^2\right]^{1/2}$.
\textcolor{blue}{\textbf{S9E2i)}} Today $a(t_0) = a_0 = 1$ then $H_0t = H_0t_0$: $t_0 = \frac{2}{3H_0}(1+f_0^2)^{1/4}\left[(1+f_0^2)^{3/4} -f_0^{3/2}\right]$
\textcolor{red}{\textbf{S10E2a)}} Heavier elements need $^8$Be (unstable). Low density of $^8$Be no C, N, O.   
\textcolor{blue}{\textbf{S10E2b)}} $B_H = 13.6$ev. For recombination to start photon temp $k_BT<B$ since \textbf{i} $k_BT$ is only mean photon energy. At any $T$ always photons with higher energies than mean $k_BT$. \textbf{ii)} Baryon-to-Photon ratio $\eta_b = 10^{-9}$: A billion more photons than protons and deutrons.
\textcolor{red}{\textbf{S10E3a)}} Let $P = \rho/3$. Continuity eq. $\dot{\rho} = -3\frac{\dot{a}}{a}(\rho + p) = -4\rho\frac{\dot{a}}{a} \rightarrow \frac{\dot{\rho}}{rho} = -4\frac{\dot{a}}{a}\rightarrow \int_{\rho_d}^{\rho(t)}\frac{d\rho}{\rho} = -4\int_{a_d}^{a(t)}\frac{da}{a} \rightarrow \ln\frac{\rho(t)}{\rho_d} = -4\ln\frac{a(t)}{a_d}\rightarrow\rho(t) = \rho_d = \left(\frac{a_d}{a(t)}\right)^4$ (*)
\textcolor{blue}{\textbf{S10E3b)}} FI., k = 0: $\frac{\dot{a}^2}{a^2} = \frac{8\pi G}{3}\rho = \frac{8\pi G}{3}\rho_d\left(\frac{a_d}{a(t)}\right)^4\rightarrow\dot{a}a = \sqrt{\frac{8\pi G}{3}\rho_d}a_d^2\rightarrow\int_{0}^{a}a'da' = \int_{0}^{t}\sqrt{\frac{8\pi G}{3}\rho_d}a_d^2dt\rightarrow \frac{a(t)}{a_d} = \left(\frac{32\pi G}{3}\rho_d\right)^{1/4}\sqrt{t}$ (**)
\textcolor{blue}{\textbf{S10E3c)}} Inserting (**) into (*): $\rho(t) = \rho_d\left(\frac{32\pi G}{3}\rho_d\right)^{-1} = \frac{3}{32\pi G}t^{-2}$.
\textcolor{blue}{\textbf{S10E3d)}} Let $\rho = N\alpha T^4 = \frac{3}{32\pi G}t^{-2}\rightarrow t(T) = \left(\frac{32\pi}{3}GN\alpha T^4\right)^{-1/2}$
\textcolor{blue}{\textbf{S10E3e)}} For any $T$, $t(T)\propto\frac{1}{\sqrt{N}}\rightarrow t$ decreasing with increasing $N$.
\textcolor{blue}{\textbf{S10E3f)}} Increaing $t(10^9K)$ makes BBN end later. Since neutrons decay, $\rho_n$ decreases exponentially until BBN ends with neutrons found in $^4$He (+some other). Thus larger $t(10^9K)$ gives neutrons more time to decay, lowering $\rho_n/\rho_p$.
\textcolor{blue}{\textbf{S10E3g)}} Almost all neutrons end up in $^4$He. For every $^4$He there are 2 neutrons. Thus $\frac{\rho_{He-4}}{\rho_p} = \frac{1}{2}\frac{\rho_n}{\rho_p}$. So  $\frac{\rho_{He-4}}{\rho_p} \propto\frac{\rho_n}{\rho_p}$ abundance of $^4$He changes proportionally to $\rho_n/\rho_p$.
\textcolor{red}{\textbf{S10E4a)}} We have $n_B\equiv\frac{\rho_b}{m_b}\approx\frac{\rho_b}{m_p} = \frac{\rho_b(\frac{a}{a_0})^{-3}}{m_p} = \frac{\Omega_{b0}(\frac{a}{a_0})^{-3}}{m_p}\rho_{c0}$.
\textcolor{blue}{\textbf{S10E4b)}} $\frac{n_e\sigma_T c}{H} = \frac{X_en_b\sigma_Tc}{H} = \frac{X_e\sigma_Tc}{H}\frac{\Omega_{b0}(\frac{a}{a_0})^{-3}}{m_p}\rho_{c0} = \frac{X_e\Omega_{b0}H_0}{H}\left(\frac{a}{a_0}\right)^{-3}\frac{\sigma_Tc}{m_p}\frac{3H_0}{8\pi G}\approx 0.0692 X_e\Omega_{b0}h(\frac{a}{a_0})^{-3}\frac{H_0}{H}$, for $m_p = 1.673\cdot 10^{-27}$kg, $G = 6.67\cdot10^{-11}m^3kg^{-1}s^{-2}$ and $H_0 = 100h$kgs$^{-1}$Mpc$^{-1}$.
\textcolor{blue}{\textbf{S10E4c)}} Assuming $k =0$ and $\Omega_{m0}  + \Omega_{r0} = 1$, then $\frac{H^2}{H_0^2} = \Omega_{r0}(1+z)^4 + \Omega_{m0}(1+z)^3$. Thus $\frac{n_e\sigma_Tc}{H} = 0.0692 X_e\Omega_{b0}h(1+z)^{3}\left(\Omega_{r0}(1+z)^4 + \Omega_{m0}(1+z)^3\right)^{-1/2} = \frac{0.0692 X_e\Omega_{b0}h(\frac{a}{a_0})^{3/2}}{\sqrt{\Omega_{m0}}\sqrt{\frac{\Omega_{r0}}{\Omega_{m0}}(1+z)}+1} = 0.0692X_e\frac{\Omega_{b0}h^2}{\sqrt{\Omega_{m0}h^2}}\left(1+\frac{1+z}{1+z_{eq}}\right)^{-1/2}$. Here $1+z_{eq} = \frac{\Omega_{r0}}{\Omega_{m0}}$ is redshift at radiation-matter equality. We let $1+z = 1000$, $1+z_{eq} \approx3500$, $\Omega_{b0}h^2\approx0.02$, $\Omega_{m0}h^2 \approx0.3*0.7^2\approx0.15$, then $\frac{n_e\sigma_Tc}{H}\approx100X_e$. When this ratio, meaning the reaction rate, drops below 1 photons decouple. This happens if $X_e<10^{-2}$.
\textcolor{red}{\textbf{S11E1a)}} Flatness problem: Universe started with special init. cond. density only deviates $\sim 10^{-61}$ from $\rho_{c0}$. Seems to special. We take a look at FI:
$\left(\frac{\dot{a}}{a}\right)^2 + \frac{kc^2}{a^2} = \frac{8\pi G}{3}\rho$. Let the critical density $\rho_c(t) = \frac{3H^2}{8\pi G}$. Inserting into FI we obtain $1 - \frac{\rho(t)}{\rho_c(t)} = -\frac{kc^2}{a^2H^2}$. Giving $\Omega(t) - 1 = \frac{kc^2}{a^2H^2}$, for $\Omega = \frac{\rho(t)}{\rho_c(t)}$. Since $(aH)^{-2}$ is an increasing quantity in matter- or radiation-dominated times, $\Omega(t) - 1$ must be increasing with time too. Therefore $\Omega(t)$ was very close to 1 early on. Today it is of order 1.
\textcolor{blue}{\textbf{S11E1b)}} We let the potential of the scalar field be $V(\phi) = \lambda \phi^p$. The slow-roll parameter $\epsilon$ is defined as $\epsilon = \frac{E_P^2}{16\pi}\left(\frac{V'}{V}\right)^2 = \frac{E_P^2}{16\pi}\left(\frac{\lambda p\phi^{p - 1}}{\lambda\phi^p}\right)^2 = \frac{E_P^2}{16\pi}\frac{p^2}{\phi^2}$. The slow-roll parameter $\eta$ is defined as 
$\eta = \frac{E_P^2}{8\pi}\frac{v''}{V} = \frac{E_P^2}{8\pi}\frac{\lambda(p-1)p\phi}{\lambda\phi^p} = \frac{E_P^2}{8\pi}\frac{(p-1)P}{\phi^2}$.
\textcolor{blue}{\textbf{S11E1c)}} Inflation ends when $\epsilon = 1$. Using the above eq. we get
$\epsilon = \frac{E_P^2}{16\pi}\frac{p^2}{\phi^2_\text{end}} = 1$, thus $\phi_\text{end} = \frac{E_P P}{\sqrt{16\pi}}$.
\textcolor{blue}{\textbf{S11E1d)}} We need 60 e-foldings to solve the flatness-problem. The total number of e-foldings is $N_\text{tot} = \frac{8\pi}{E_P^2}\int_{\phi_\text{end}}^{\phi_i}\frac{V}{V'}d\phi = \frac{8\pi}{E_P^2}\int_{\phi_\text{end}}^{\phi_i}\frac{\lambda \phi^p}{\lambda p \phi^{p-1}}d\phi = \frac{8\pi}{E_P^2}\int_{\phi_\text{end}}^{\phi_i}\frac{\phi}{p}d\phi = \frac{4\pi}{E_P^2 p}(\phi_i^2 - \phi_\text{end}^2) = 60$ which gives $\phi_i = \sqrt{\frac{15}{\pi}E_P^2p + \phi_\text{end}^2} = \sqrt{\frac{15}{\pi}E_P^2p + \frac{E_P^2p^2}{16\pi}} = E_P\sqrt{\frac{p}{4\pi}(60 + \frac{p}{4})}$.
\textcolor{red}{\textbf{S11E2a)}} Horizon problem:
CMB same temp 2.73K everywhere, only fluctuates $\sim10^{-5}K$. Same $T$ at two points separated by more than the part. horizon at the time. Not causally connected, no physical process to cause it. 
\textcolor{blue}{\textbf{S11E2b)}} The density and pressure of the field are given by
$\rho = \frac{1}{2}\dot{\phi}^2 + V(\phi)$, $P = \frac{1}{2}\dot{\phi}^2 - V(\phi)$. If $V = 0$ then $\rho = P$. Then FII becomes
$\ddot{a} = -\frac{4\pi G}{3}\left(\rho + 3P\right)a = -\frac{4\pi G}{3}\left(4\rho\right)a <0$ Thus there is no accelerated expansion, and hence no inflation.
\textcolor{blue}{\textbf{S11E2c)}} If $\dot{\phi}^2 = 2V$ then the pressure and density 
$\rho = 3V$, $P = 0$. This is the same as for non-relat. matter making the universe behave similar to an EdS model. No inflation.
\textcolor{blue}{\textbf{S11E2d)}} Let $V = V_0 e^{-\phi/E_P}$, then the slow-roll parameters are $\epsilon = \frac{E_P^2}{16\pi}\left(\frac{V'}{V}\right)^2 = \frac{E_P^2}{16\pi}\left(\frac{-V_0/E_P e^{-\phi/E_P}}{V_0 e^{-\phi/E_P}}\right)^2 = \frac{1}{16\pi}<1$ and $\eta = \frac{E_P^2}{8\pi}\left(\frac{V''}{V}\right) = \frac{E_P^2}{8\pi}\left(\frac{V_0/E_P^2 e^{-\phi/E_P}}{V_0e^{-\phi/E_p}}\right) = \frac{1}{8\pi} <1$. Because $\epsilon<1$ and $\eta<1$ we have inflation, but they are constant and will remain as they are, so inflation will never stop.
\textcolor{red}{\textbf{S11E3a)}} Let $V = \lambda|\phi|$, $\lambda>0$. Slow-roll parameters:
$\epsilon = \frac{E_P^2}{16\pi}\left(\frac{V'}{V}\right)^2 = \frac{E_P^2}{16\pi}\frac{\lambda^2}{\lambda^2|\phi|^2}\left(\dv{|\phi|}{\phi}\right)^2 = \frac{E_P^2}{16\pi}\frac{\phi^2}{|\phi|^4} = \frac{E_p^2}{16\pi}\frac{1}{\phi^2}$ and $\eta = \frac{E_p^2}{8\pi}\frac{V''}{V} = \frac{E_p^2}{8\pi}\frac{1}{|\phi|}\left(\frac{1}{|\phi|}+\phi\dv{\phi}\frac{1}{|\phi|}\right) = \frac{E_p^2}{8\pi}\left(\frac{1}{|\phi|^2}-\frac{1}{|\phi|^2}\right) = 0$.  Inflation ends at $\epsilon = \epsilon(\phi_\text{end}) = 1\rightarrow\frac{E_p^2}{16\pi}\frac{1}{\phi_\text{end}^2} = 1\rightarrow \phi_{end} = \frac{E_p}{4\sqrt{\pi}}$.
\textcolor{blue}{\textbf{S11E3b)}} Using SLA: $H^2\approx \frac{8\pi G}{3c^2}V(\phi)$ and $3H\dot{\dot{\phi}}\approx-\hbar c^3V'(\phi)$ giving $\frac{\dot{\phi}}{H} = -\frac{\hbar c^5}{8\pi G}\frac{V'(\phi)}{V(\phi)} = -\frac{E_p^2}{8\pi}\frac{V'}{V}$. For 60 e-foldings we get $60 = \frac{8\pi}{E_p^2}\int_{\phi_\text{end}}^{\phi_i}\frac{V}{V'}d\phi = \frac{8\pi}{E_p^2}\frac{|phi|^2}{\phi}d\phi = \frac{4\pi}{E_p^2}(\phi_i^2 + \phi_\text{end}^2)\rightarrow \phi_i = \sqrt{\frac{E_p^2}{4\pi}\frac{241}{4}} = \frac{\sqrt{241}}{4\sqrt{\pi}}E_p = \sqrt{241}\phi_\text{end}$.
\textcolor{blue}{\textbf{S11E3c)}} Tensor-to-scalar ratio $r = 3\sqrt{\epsilon_*}$. For $\epsilon_* = \epsilon(\phi_*)$, field value $\phi_*$ when 50-efoldings left. Find $\phi_*$ by $50 = \frac{8\pi}{E_p^2}\int_{\phi_\text{end}}^{\phi_*}\frac{V}{V'}d\phi = \frac{8\pi}{E_p^2}\int_{\phi_\text{end}}^{\phi_*}\phi d\phi = \frac{4\pi}{E_p^2}(\phi_*^2 - \phi_\text{end}^2)\rightarrow\phi_*^2 = \frac{25E_p^2}{2\pi} + \phi_\text{end}^2 = \frac{E_p^2}{16\pi}201 = 201\phi_\text{end}^2\rightarrow\epsilon_* = \frac{\phi_{end}^2}{\phi_*^2} = \frac{1}{201}\rightarrow r = \frac{3}{\sqrt{201}}\approx 0.21$.
\textcolor{red}{\textbf{S11E4a)}} Consider FI. $H^2 = \frac{8\pi G}{3}\rho_\phi =  \frac{1}{2M_P^2}(V + \frac{1}{2}\dot{\phi})$ and the continuity eq. $\dot{\rho_\phi} = -3H(\rho_\phi + P_\phi)$ gives $\ddot{\phi}\dot{\phi} + \dot{\phi}V' = -3H(\frac{1}{2}\dot{\phi}^2 + V + \frac{1}{2}\dot{\phi}^2 - V)$ Which gives $\ddot{\phi} + 3H\dot{\phi} = -V'(\phi)$.
\textcolor{blue}{\textbf{S11E4b)}} We differentiate $H^2$: $\dv{H^2}{\phi} = 2H H' = \frac{1}{3M_P^2}\left(\dv{V}{\phi} + \frac{1}{2}\dv{\dot{\phi}^2}{\phi}\right)
= \frac{1}{3M_P^2}\left(\dv{V}{\phi} + \dot{\phi}\frac{\ddot{\phi}}{\dot{\phi}}\right)$. Inserting for $V'(\phi)$ from previous task we get $\dot{\phi} = -2M_P^2H'$.
\textcolor{blue}{\textbf{S11E4c)}} Consider FI: 
$H^2(\phi) = \frac{8\pi G}{3}\rho_\phi = \frac{1}{3M_P^2}\left(\frac{1}{2}\dot{\phi}^2 + V\right) =  \frac{1}{3M_P^2}\left(2M_P^4(H'(\phi))^2 + V\right)$ which gives $(H'(\phi))^2 -\frac{3}{2M_P^2}H^2 = -\frac{V}{2M_P^4}$.
\textcolor{blue}{\textbf{S11E4d)}} Consider the linear homogeneous perturbation $H(\phi) = H_0(\phi) + \delta H(\phi)$. Then 
$(H_0'(\phi) + \delta H'(\phi))^2 -\frac{3}{2M_P^2}(H_0(\phi) + \delta H(\phi))^2 = -\frac{V}{2M_P^4}$. Only keeping first order terms we get and that $H_0$ is a solution $H_0'\delta H - \frac{3}{2M_p^2}H_0\delta H = 0$.
\textcolor{blue}{\textbf{S11E4e)}} $H_0'\delta H - \frac{3}{2M_p^2}H_0\delta H = 0$ gives $\frac{\delta H'}{\delta H} = \frac{3}{2M_P^2}\frac{H_0}{H_0'}$. Integrating $\int_{\delta H(\phi_i)}^{\delta H(\phi)}\dv{(\delta H)}{\delta H} = \frac{3}{2M_P^2}\int_{\phi_i}^{\phi}\frac{H_0}{H_0'}d\phi \rightarrow \ln(\frac{\delta H(\phi)}{\delta H(\phi_i)}) = \frac{3}{2M_P^2}\int_{\phi_i}^{\phi}\frac{H_0}{H_0'}d\phi \rightarrow \delta H(\phi) = \delta H(\phi_i)\exp(\frac{3}{2M_P^2}\int_{\phi_i}^{\phi}\frac{H_0}{H_0'}d\phi)$
Since we only consider expanding universes $\dot{\phi}>0$ and $H_0>0$, $\dot{\phi} = -2M_P^2H'$ implies that $H'(\phi)<0$ making 
$\frac{3}{2M_P^2}\int_{\phi_i}^{\phi}\frac{H_0}{H_0'}d\phi<0$. The perturbation $\delta H(\phi)\sim e^{-x}$ and vanishes exponentially fast.
\textcolor{red}{\textbf{S11E6}} Let $V = \lambda \phi^p$, $\lambda, p>0$. Slow-roll parameters $\epsilon = \frac{E_P^2}{16\pi}\left(\frac{V'}{V}\right)^2 = \frac{E_P^2}{16\pi}\frac{p^2}{\phi^2}$ and $\eta = \frac{E_P^2}{8\pi}\frac{V''}{V} = \frac{E_P^2}{8\pi}\frac{p(p-1)}{\phi^2}$. We see that $\epsilon = \frac{p}{2(p-1)}\eta$, meaning that they are of the same magnitude. The slow-roll condition $\epsilon\ll1$ and $|\eta|\gg1$ thus reduces to the former. Thus $\phi\ll\frac{E_Pp}{4\sqrt{\pi}} = \phi_\text{end}$Inflation starts at $\phi_i\gg\phi_\text{end}$. The total number of e-folding is then $N_\text{tot} = \frac{8\pi}{E_P^2}\int_{\phi_\text{end}}^{\phi_i}\frac{V}{V'}d\phi = \frac{8\pi}{E_P^2p}\int_{\phi_\text{end}}^{\phi_i}\phi d\phi = \frac{4\pi}{E_P^2p}(\phi_i^2 - \phi_\text{end}^2) = \frac{4\pi}{E_P^2p}\phi_\text{end}^2(\frac{\phi_i^2}{\phi_\text{end}^2} - 1)$. In most realistic models $p\sim1$. If $\epsilon\ll1$ then $\phi_i\gg\phi_\text{end}$ thus $N_\text{tot}\approx\frac{\phi_i^2}{\phi_\text{end}^2}\gg 1$. This potential causes the universe to expand a lot.
\textcolor{red}{\textbf{S11E7a)}} $V(\phi) = V_0 \exp(-\lambda\phi)$ and we let $\hbar = c = 1$. In the SRA, we get the two eq. $H^2 \approx \frac{8\pi G}{3c^2}V(\phi) = \frac{V}{3M_p^2}$, for Planck mass $M_P = 1/\sqrt{8\pi G} = E_P^2/\sqrt{8\pi}$, and $3H\dot{\phi}\approx-\hbar c^3V'(\phi) = -V'$ in natural units.
\textcolor{blue}{\textbf{S11E7b)}} This gives $H = \sqrt{\frac{V}{3M_P^2}} = \sqrt{\frac{V_0}{3M_p^2}}\exp(-\frac{\lambda\phi}{2})$. Inserting into second eq. we get $3\dot{\phi}\sqrt{\frac{V_0}{3M_P^2}}\exp(-\lambda\phi/2) = -\lambda\exp(-\lambda\phi)$ which gives $\exp(\lambda\phi/2)d\phi = \lambda/3 \sqrt{3V_0M_P^2}dt$. Since the potential $V$ is decreasing exponentially with increasing $\phi$, it reaches its minimum of $V = 0$ at $\phi\to\infty$. We thus need to start at $\phi\to-\infty$ in the lower limit, which we for simplicity let coincide with $t\to0$. Thus we get $\int_{-\infty}^{\phi(t)}e^{\lambda\phi/2}d\phi = \int_{0}^{t}\sqrt{\lambda^2M_P^2V_0/3}dt$ giving $2e^{\lambda\phi(t)/2}/\lambda = \sqrt{\lambda^2M_P^2V_0/3}t$ resulting in $\phi(t) = 2/\lambda\ln(\frac{\lambda^2M_P}{2}\sqrt{\frac{V_0}{3}}t)$. Inserting back into the other eq. 
$(\dot{a}/a)^2 = V/3M_P^2 = \frac{V_0}{3M_P^2}\exp(2\ln((\frac{\lambda^2M_P}{2}\sqrt{\frac{V_0}{3}}t))) = \frac{V_0}{3M_P^2}\left(\frac{\lambda^2M_P}{2}\sqrt{\frac{V_0}{3}}t\right)^{-2} = 4/(M_P^2\lambda^4 t^2)$ giving $da/a = 2dt/(M_P^2\lambda^2 t)$ giving $\ln a(t) = 2/(M_P^2\lambda^2)\ln t + \ln C$ resulting in $a(t) = Ct^{2/M_p^2\lambda^2}$, for a constant $C$. The behaviour of $a$  of the model is the reason why it is called power-law-inflation. Inflation requires $\ddot{a}>0$, and for power law $a\propto t^n$, we must have $n>1$. Thus $2/(\lambda^2M_P^2)>1$ giving $\lambda < \sqrt{2}/(\lambda M_P)$.
\textcolor{blue}{\textbf{S11E7c)}} Ansatz $a(t) = Ct^\alpha$ and $\phi(t) = 2/\lambda \ln(Bt)$. From these we get $\dot{a} = C\alpha t^{\alpha - 1}$, $\dot{\phi} = 2/(\lambda t)$, $\ddot{\phi} = -2/(\lambda t^2)$ and $V(\phi) = V_0 e^{-\lambda\phi} = V_0/(Bt)^2 = -V'(\phi)/\lambda$. Inserting into full ode we get two eq. $\frac{2}{\lambda t^2} + \frac{6\alpha}{\lambda t^2} - \frac{\lambda V_0}{B^2t^2} = 0$ and $H^2 = \frac{\alpha^2}{t^2} = \frac{V_0}{2M_P^2 B^2t^2} + \frac{2}{3M_P^2\lambda^2 t^2}$. These should for all times $t$, thus 
(I) $\frac{6\alpha}{t^2} - \frac{\lambda V_0}{B^2} = \frac{2}{\lambda}$ and (II) $\alpha^2 - \frac{V_0}{3M_P^2B^2} = \frac{2}{3M_P^2 \lambda^2M_P^2}$. Multiplying the first by $-\frac{1}{3M_P^2\lambda}$ and then adding (I) + (II) we get $\alpha^2 - \frac{2}{M_P^2\lambda^2}\alpha = 0$. Thus we get two solutions, $\alpha = 0$ which makes $a(t) = Ct^2 = \text{constant}$ which is not interesting, and $\alpha = \frac{2}{M_P^2\lambda^2}$ which is the same as in SRA! We substitute this solution into (I): $\frac{6}{\lambda}\frac{2}{M_P^2\lambda^2} - \frac{\lambda V_0}{B^2} = \frac{2}{\lambda}$ giving 
$\frac{\lambda V_0}{B} = \frac{12}{M_P^2\lambda^3} - \frac{2}{\lambda} = \frac{12 - 2M_P^2\lambda^2}{M_P^2\lambda^3}$, Thus the constant $B^2 = \frac{\lambda^2 V_0}{2}\frac{\lambda^2M_P^2}{6-M_P^2\lambda^2}$. In the slow-roll limit $\epsilon\ll 1$ making $\epsilon = \frac{M_P^2}{2}\left(\frac{V'}{V}\right)^2 = \frac{M_p^2}{2}\left(\frac{-\lambda V_0 e^{-\lambda\phi}}{V_0 e^{-\lambda\phi}}\right) = \frac{M_p^2}{2}\lambda^2 = \text{constant}$. Thus $M_P^2\lambda^2 \ll 1$. Thus $B^2 \approx \frac{\lambda V_0}{2}\frac{\lambda^2M_P^2}{6} = \frac{\lambda^4M_P^2}{4}\frac{V_0}{3}$ making $B\approx \frac{\lambda^2M_P}{2}\sqrt{\frac{V_0}{3}}$, which we recognize from SRA. The general solutions are thus 
$a(t) = Ct^\alpha = Ct^{2/\lambda^2M_P^2} = Ct^{1/\epsilon}$ and $\phi(t) = \frac{2}{\lambda}\ln\left[\lambda^2 M_P\sqrt{\frac{V_0}{2(6-M_P^2\lambda^2)}}t\right]$
\textcolor{blue}{\textbf{S11E7d)}} The problem is that $\epsilon = \text{constant}$ in this model. Slow-roll inflation starts only if $\epsilon \ll 1$. But the inflation cannot stop.
\textcolor{blue}{\textbf{S11E7e)}}
The tensor-to-scalar ration is given as 
$r = 16 \epsilon_*$, where $*$ denotes that $\epsilon$ be evaluated at the time when observationally relevant length scales crossed the horizon during inflation. However, in this model $\epsilon$ is a constant, thus $r = 16 \frac{1}{2}M_P^2\lambda^2 = 8M_P^2\lambda^2$.
\end{multicols}
} } }
\end{document}